\begin{enumerate}
  \item
    \begin{itemize}
      \item   
        $\Rd{H,K}$ est reflexive:
        Pour tout $x\in G$, on a $x = exe$, donc $x \Rd{H,K} x$.

      \item
        $\Rd{H,K}$ est symétrique:
        Soit $(x,y)\in G^2$ tel que $x \Rd{H,K} y$. Il existe $(h,k)\in
        H\times K$ tel que $y = hxk$. On en déduit que $x = h^{-1}yk^{-1}$ où
        $(h^{-1},k^{-1})\in H\times K$, donc $y \Rd{H,K} x$.

      \item
        $\Rd{H,K}$ est transitive:
        Soit $(x,y,z)\in G^3$ tel que $x \Rd{H,K} y$ et $y \Rd{H,K} z$. Il
        existe $(h_1,k_1)\in H\times K$ et $(h_2,k_2)\in H\times K$ tels que
        $y = h_1xk_1$ et $z = h_2yk_2$. On en déduit que $z = h_2h_1xk_1k_2$ où
        $(h_2h_1,k_1k_2)\in H\times K$, donc $x \Rd{H,K} z$.
    \end{itemize}

    Nous avons montré que la relation binaire $\Rd{H,K}$ est une relation
    d'équivalence sur $G$. La classe d'équivalence de tout $x\in G$ est
    \[
      \cl{x} = \setm{hxk}{(h,k)\in H\times K} = HxK.
    \]

  \item
    \begin{itemize}
      \item
        $\lambda$ est injective; en effet, supposons
        $\lambda(hxk) = \lambda(h'xk')$, c'est-à-dire
        $x^{-1}hxk = x^{-1}h'xk'$. On a alors $hxk = h'xk'$. 

      \item
        $\lambda$ est surjective; il n'y a rien à faire.
    \end{itemize}

    En conclusion, $\lambda$ est une bijection.

  \item
    \begin{enumerate}
      \item[$\alpha$)]
        À la question précédente, nous avons montré que les ensembles 
        $Hx_iK$ et $x_i^{-1}Hx_iK$ sont équipotents, donc 
        $\card{Hx_iK} = \card{x_i^{-1}Hx_iK}$.

      \item[$\beta$)]
        $x_i^{-1}Hx_i$ est l'image du sous-groupe $H$ de $G$ par l'automorphisme
        intérieur $G\to G$, $g\mapsto x_i^{-1}gx_i$, il s'ensuit que
        $x_i^{-1}Hx_i$ est un sous-groupe de $G$ et que
        $\ordre(x_i^{-1}Hx_i)=\ordre(H)$.

      \item[$\gamma$)]
        Pour tout $i$ $(1\leq i\leq r)$, $x_i^{-1}Hx_i$ et $K$ sont deux
        sous-groupes finis. Appliquons leurs la formule de l'exercice 3.
        Il vient
        \[
          \card{x_i^{-1}Hx_iK} 
          = \frac{\ordre(x_i^{-1}Hx_i)\ordre(K)}
          {\ordre(x_i^{-1}Hx_i\cap K)},
        \]
        puis, en utilisant les propriétés $\alpha$) et $\beta$),
        \[
          \card{Hx_iK} = \frac{\ordre(H)\ordre(K)}{\ordre(x_i^{-1}Hx_i\cap K)}.
        \]
    \end{enumerate}
      
    Les classes doubles de $G$ modulo $H$ et $K$ constituent une
    partition du groupe $G$, donc
    \[ 
      \ordre(G) = \sum_{i=1}^r \card{Hx_iK}.
    \]
    D'après propriété $\gamma)$, cette égalité devient
    \[
      \ordre(G) = \ordre(H)\ordre(K)\sum_{i=1}^r d_i^{-1}
      \quad\text{où}\quad d_i = \ordre(x_i^{-1}Hx_i\cap K).
    \]

  \item
    Soit $H=\Gr{\tau_1} = \set{e,\tau_1}$ et $K=\Gr{\tau_2} = \set{e,\tau_2}$.
    Déterminons les classes doubles de $\Sym{3}$ modulo $H$ et $K$. Nous nous
    aiderons de la table de Cayley de $\Sym{3}$ (exemple 1.18). On
    trouve deux classes:
    \[
      HeK = \set{e, \tau_1, \tau_2, \sigma_1}
      \quad\text{et}\quad
      H\tau_3 K = \set{\tau_3,\sigma_2}.
    \]
    Cet exemple montre que deux classes doubles distinctes ne sont pas, en
    général, équipotentes.

    Déterminons, à présent, les classes doubles modulo $K$ et $H$. On trouve
    encore deux classes:
    \[
      KeH = \set{e, \tau_1, \tau_2, \sigma_2}
      \quad\text{et}\quad
      K\tau_3 H = \set{\tau_3,\sigma_1}.
    \]
    On remarque que $\set{KeH, H\tau_3K} \neq \set{HeK, K\tau_3 H}$, ce qui
    montre, qu'en général, dans un groupe non abélien, $\Rd{H,K}\neq
    \Rd{K,H}$.
\end{enumerate}
