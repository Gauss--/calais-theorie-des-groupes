Les sous-groupes $H$ et $K$ étant d'indices finis dans $F$, d'après le théorème
de Poincaré (théorème 2.17) il en est de même du sous-groupe $H\cap K$. 
La formule des indices (théorème 2.18) nous donne alors
\[
  [F:H\cap K] = [F:H][H:H\cap K]
  \quad\text{et}\quad
  [F:H\cap K] = [F:K][K:H\cap K].
\]
Nous avons donc l'égalité
\[
  [F:H][H:H\cap K] = [F:K][K:H\cap K].
\]
Comme $[F:H]$ et $[F:K]$ sont premiers entre eux, le théorème de Gauss affirme
que
\[
  [F:H] \divise [K:H\cap K]
  \quad\text{et}\quad
  [F:K] \divise [H:H\cap K],
\]
d'où les inégalités
\[
  [F:H] \leq [K:H\cap K]
  \quad\text{et}\quad
  [F:K] \leq [H:H\cap K].
\]
Dans l'exercice 2, nous avons montré que
\[
  [K:H\cap K]\leq [F:H]
  \quad\text{et}\quad
  [H:H\cap K]\leq [F:K].
\]
Finalement, nous déduisons des inégalités précédentes que
\[
  [F:K] = [H:H\cap K]
  \quad\text{et}\quad
  [F:H] = [K:H\cap K].
\]
