\begin{enumerate}[a)]
\item On a $(0*0)*1 = (0-0)-1 = -1$ et $0*(0*1) = 0-(0-1) = 1$ donc $*$ n'est
pas associative.  De plus $0*1 = 0-1 = -1$ et $1*0 = 1-0 = 1$ donc $*$ n'est
pas non plus commutative.

\item Soit $a\in\Z$. $a*e = a$, soit $a-e = a$ implique $e = 0$.  De plus $a*0
= a-0 = a$ pour tout $a\in\Z$.  Donc $0$ est l'unique élément neutre à droite
pour $*$.  Si $a \neq 0$, on a $0*a = -a \neq a$ donc $0$ n'est pas un élément
neutre.

\item Pour tout $ a\in\Z$, on a $a*a = a-a = 0$, donc $a$ est un symétrique à
droite de $a$.  
\end{enumerate}

% TODO Si une loi interne admet un élément neutre à gauche (resp. à droite),
% celui-ci n'est pas nécessairement unique. Par exemple, soit E muni de la loi
% *:E->E, (x,y)->y, alors tout élément de E est un élément neutre à gauche. En
% revanche, si * admet un neutre à gauche et un neutre à droite, alors ils sont
% égaux, et c'est l'unique élément neutre de * (Calais, remarque 1.2).

