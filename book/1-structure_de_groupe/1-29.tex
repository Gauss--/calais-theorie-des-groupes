Il est clair que la loi de composition ainsi définie est une loi interne.
Montrons qu'elle est associative. Soit $f,g,h\in G^E$. Pour tout $x\in E$, on
a:
%
\begin{align*}
  (f(gh))(x) &= f(x)(gh)(x) \\
             &= f(x)(g(x)h(x)) \\
             &= (f(x)g(x))h(x) \\
             &= (fg)(x)h(x) \\
             &= ((fg)h)(x)
\end{align*}
%
d'où $f(gh) = (fg)h$.

Montrons l'existence d'un élément neutre.  Soit $j\in G^E$ définie pour tout
$x\in E$ par $j(x)=e$. Alors pour tout $x\in E$, on a $(fj)(x) = f(x)j(x) =
f(x) = j(x)f(x) = (jf)(x)$ d'où  $jf = f = fj$.

Reste à montrer que toute $f\in G^E$ est inversible. On définit $g\in G^E$ par
$g(x) = f(x)^{-1}$ pour tout $x\in E$. On a $(fg)(x) = f(x)g(x) = f(x)f(x)^{-1}
= e = j(x)$ \textit{i.e.} $fg = j$. On montre de même que $gf = j$.

Finalement, $G^E$ est un groupe.

On suppose que $G$ est abélien. Soit $f,g\in G^E$. Pour tout $x\in E$, on a
$f(x)g(x) = g(x)f(x)$ c'est-à-dire $fg = gf$; $G^E$ est abélien. Réciproquement,
supposons que $G^E$ est abélien. Soit $a,b\in G$, et $f,g\in G^E$ deux
applications constantes égales à $a$ et $b$ respectivement. Pour tout $x\in E$,
nous avons $ab = f(x)g(x) = (fg)(x) = (gf)(x) = g(x)f(x) = ba$. 
