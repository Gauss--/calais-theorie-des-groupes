\begin{enumerate}
  \item 
    Démontrons que $K_1$ est un sous-groupe de $\GL(2,\R)$.

    Pour tout $(a,b)\in\set{-1,1}^2$, posons 
    \[
      M_{a,b} = \mat{a & 0 \\ 0 & b}
    \]
    et notons $I_2$ la matrice unité $M_{1,1}$. Ainsi
    \[
      K_1 = \set{I_2, M_{1,-1}, M_{-1,1}, M_{-1,-1}}.
    \]
    Soit $(a,a',b,b')\in\set{-1,1}^4$. On a
    \[
      M_{a,b} M_{a',b'} = M_{aa',bb'}\quad\text{avec}\quad (aa',bb')\in\set{-1,1}^2, 
    \]
    donc 
    \[
      M_{a,b}M_{a',b'} \in K_1.
    \]
    On vérifie que $M_{a,b}^2 = I_2$, donc
    \[
      M_{a,b}^{-1} = M_{a,b}\in K_1.
    \]
    L'ensemble $K_1$ est donc un sous-groupe de $\GL(2,\R)$.

    Démontrons que $K_2$ est un groupe.  
    
    Soit $\cl{x}$ et $\cl{y}$ deux éléments de $K_2$ avec $(x,y)\in\set{1,3,5,7}^2$. 
    Le produit $xy$ est impair et $8$ est pair, donc les représentants de
    $\cl{xy}$ sont impairs. On en déduitt que $\cl{x}\cdot\cl{y} =
    \cl{xy}\in K_2$, donc le produit définit une loi interne sur $K_2$. On sait
    que $\Zn{8}$ est un anneau, donc le produit est associatif. L'élément unité
    est $\cl{1}$.  De plus, pour tout $\cl{x}\in K_2$, on a $\cl{x}^2 = \cl{1}$,
    donc $\cl{x}^{-1} = \cl{x}\in K_2$. Nous avons ainsi démontré que $K_2$ est
    un groupe multiplicatif.

  \item 
    Démontrons que les groupes $K_1$ et $K_2$ sont isomorphes.

    Écrivons les tables de Cayley des groupes $K_1$ et $K_2$.
    
    \begin{center}
      \begin{tabular}{c|cccc}
        $\times$    & $I_2$       & $M_{1,-1}$  & $M_{-1,1}$  & $M_{-1,-1}$ \\
        \midrule
        $I_2$       & $I_2$       & $M_{1,-1}$  & $M_{-1,1}$  & $M_{-1,-1}$ \\
        $M_{1,-1}$  & $M_{1,-1}$  & $I_2$       & $M_{-1,-1}$ & $M_{-1,1}$  \\
        $M_{-1,1}$  & $M_{-1,1}$  & $M_{-1,-1}$ & $I_2$       & $M_{1,-1}$  \\
        $M_{-1,-1}$ & $M_{-1,-1}$ & $M_{-1,1}$  & $M_{1,-1}$  & $I_2$
      \end{tabular}
    \end{center}
    \begin{center}
      \begin{tabular}{c|cccc}
        $\times$ & $\cl{1}$ & $\cl{3}$ & $\cl{5}$ & $\cl{7}$ \\
        \midrule
        $\cl{1}$ & $\cl{1}$ & $\cl{3}$ & $\cl{5}$ & $\cl{7}$ \\
        $\cl{3}$ & $\cl{3}$ & $\cl{1}$ & $\cl{7}$ & $\cl{5}$ \\
        $\cl{5}$ & $\cl{5}$ & $\cl{7}$ & $\cl{1}$ & $\cl{3}$ \\
        $\cl{7}$ & $\cl{7}$ & $\cl{5}$ & $\cl{3}$ & $\cl{1}$
      \end{tabular}
    \end{center}

    Renommons les éléments de la table de Cayley du groupe $K_1$ à l'aide de la
    bijection $\varphi\colon K_1\to K_2$ définie par
    \[
      \varphi(I_2) = \cl{1},\quad 
      \varphi(M_{1,-1}) = \cl{3}, \quad
      \varphi(M_{-1,1}) = \cl{5} \quad\text{et}\quad
      \varphi(M_{-1,-1}) = \cl{7}.
    \]
    En d'autres termes, remplaçons $x$ par $\varphi(x)$ pour tout $x\in K_1$.
    Nous obtenons ainsi la même table de Cayley que celle de $K_2$. Cela nous
    permet de conclure que les groupes $K_1$ et $K_2$ sont isomorphes.

    Démontrons que $K_1$ et $K_2$ sont isomorphe au groupe de Klein.

    Étant donné que $K_1$ et $K_2$ sont isomorphes, il suffit de démontrer que
    $K_2$ est isomorphe au groupe de Klein.  Posons $H_1 = \set{\cl{1},\cl{3}}$
    et $H_2 = \set{\cl{1},\cl{5}}$. Ce sont deux sous-groupes de $K_2$ tel que
    $H_1\iso\Zn{2}$ et $H_2\iso\Zn{2}$.  Comme
    $\cl{3}\times\cl{5}=\cl{7}$, on a $K_2 = H_1H_2$. 
    De plus $H_1\cap H_2 = (\cl{1})$. La proposition 1.85 nous permet de
    conclure que
    \[
      K_2\iso\Zn{2}\times\Zn{2}.
    \]
\end{enumerate}
