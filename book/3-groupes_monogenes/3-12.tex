\begin{enumerate}
  \item 
    Soit 
    \[
      M = \matrice{a & b \\ 0 & c}
      \quad\text{et}\quad 
      N = \matrice{a' & b' \\ 0 & c'}
    \]
    deux matrices de $G$. On trouve
    \[
      MN^ {-1} = \matrice{a & b \\ 0 & c} 
                  \dfrac{1}{a'c'} \matrice{ c' & -b' \\ 0 & a'}
               = \matrice{ac' & -ab'+ba' \\ 0 & a'c}
    \]
    avec $(ac')(a'c) = (ac)(a'c')\neq 0$, donc $MN^{-1}\in G$
    et $G$ est un sous-groupe de $\GL(2,\R)$.

    L'application 
    \[
      \R\to G,\quad a\mapsto\matrice{a & 0 \\ 0 & a}
    \]
    est injective, donc le groupe $G$ est infini.
   
    Les matrices 
    \[
      A = \matrice{1 & -1 \\ 0 & 2}
      \quad\text{et}\quad 
      B = \matrice{1 & 2 \\ 0 & 1}
    \]
    sont des éléments de $G$ tels que
    \begin{gather*}
      AB = \matrice{1 & 1 \\ 0 & 2} 
      \quad\text{et}\quad
      BA = \matrice{1 & 3 \\ 0 & 2},
    \end{gather*}
    donc le groupe $G$ n'est pas abélien.

  \item
    L'application $\varphi\from \R \to G$ définie par 
    \[
      \varphi(x) = \matrice{1 & x \\ 0 & 1}
    \]
    est un homomorphisme de groupes; en effet, pour tout $(x,x')\in\R^2$,
    \[
      \varphi(x+x') 
      = \matrice{1 & x+x' \\ 0 & 1} 
      = \matrice{1 & x \\ 0 & 1}\matrice{1 & x' \\ 0 & 1} 
      = \varphi(x)\varphi(x').
    \]
    Son image est $H$, donc $H$ est un sous-groupe de $G$.  De plus,
    l'homomorphisme $\varphi$ est injectif, donc d'après le 1\up{er} théorème
    d'isomorphisme, nous en déduisons que le groupe $(\R,+)$ est isomorphe au
    groupe $H$.

    \item 
      Les éléments d'ordre $2$ du groupe $G$ sont les matrices
      $\matrice{a & b \\ 0 & c}$ telles que
      \[
        \matrice{a & b \\ 0 & c}^2 
        = \matrice{a^2 & ab+bc \\ 0 & c^2} 
        = \matrice{1 & 0 \\ 0 & 1}
        \quad\text{et}\quad
        \matrice{a & b \\ 0 & c}\neq\matrice{1 & 0 \\ 0 & 1},
      \]
    donc les coefficients $a$, $b$ et $c$ sont les solutions du système
    \[
        \begin{cases}
          a \in\{\pm 1\}\\
          c \in\{\pm 1\}\\
          b(a+c) = 0\\
          (a,b,c) \neq (1,0,1)
        \end{cases}
    \]
    soit
    \[
      \begin{cases}
        a = 1\\
        b = -1\\
        b \in \R
      \end{cases}
      \text{ou}\quad
      \begin{cases}
        a = -1\\
        c = 1\\
        b \in \R
      \end{cases}
      \text{ou}\quad
      \begin{cases}
        a = -1\\
        c = -1\\
        b = 0
      \end{cases}
    \]
    Les éléments d'ordre 2 de $G$ sont les matrices
    \[
      \matrice{-1 & 0 \\ 0 & -1}
      \quad\text{et}\quad
      \setm{\matrice{1 & b \\ 0 & -1}, \matrice{-1 & b\\ 0 & 1}}{b\in\R}.
    \]
    Les matrices 
    \[
      \matrice{1 & 2 \\ 0 & -1}
      \quad\text{et}\quad 
      \matrice{1 & 3 \\ 0 & -1}
    \]
    sont des éléments d'ordre 2, mais leur produit
    \[
      \matrice{1 & 2 \\ 0 & -1} \matrice{1 & 3 \\ 0 & -1} 
      = \matrice{1 & 1 \\ 0 & 1}
    \]
    est un élément d'ordre infini; en effet, on démontre par récurrence que
    pour tout entier $n>0$,
    \[
      \matrice{1 & 1 \\ 0 & 1}^n = 
        \matrice{1 & n \\ 0 & 1} \neq \matrice{1 & 0 \\ 0 & 1}.
    \]
\end{enumerate}


