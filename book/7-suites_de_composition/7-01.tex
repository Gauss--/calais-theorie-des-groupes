\begin{enumerate} 
  \item 
    Soit $K$ un sous-groupe normal minimal de $G$. On sait que
    $\ZG(K)\caracteristique K$ (proposition 4.43) donc $\ZG(K)\normal G$
    (proposition 4.44).  La minimalité de $K$ implique $\ZG(K) = (e)$ ou
    $\ZG(K) = K$.

  \item
    Commençons par montrer que $KL$ est un sous-groupe de $G$.
    De $K\normal G$, on déduit $LKL^{-1}\subseteq K$, puis $LK\subseteq KL$.
    De même, $L\normal G$ implique $KL\subseteq LK$, si bien que $LK = KL$.
    La proposition~1.47 permet de conclure.

    Montrons que l'application
    \[
      \varphi\from K\times L\to KL, 
      (k,l) \mapsto kl
    \]
    est un isomorphisme de groupes.

    Vérifions que $kl = lk$ pour tout $(k,l)\in K\times L$. Nous savons
    que 
    \[
      K\normal G\text{ et } L\normal G
      \implies K\cap L\normal G\text{ et } K\cap L\leq K.
    \]
    La minimalité de $K$ implique que $K\cap L = (e)$ ou $K\cap L = K$.
    Supposons que $K\cap L = K$. On a alors $K\subseteq  L$.  La minimalité de
    $L$ et $K\neq L$ impliquent que $K = (e)$, puis $K\cap L = (e)$.  On en
    déduit que si $k\in K$ et $l\in L$ alors
    \[
      klk^{-1}l^{-1} = k(lk^{-1}l^{-1})\in K
      \quad\text{et}\quad
      klk^{-1}l^{-1} = (klk^{-1})l^{-1}\in L,
    \]
    c'est-à-dire $klk^{-1}l^{-1}\in L\cap K = (e)$, donc $kl = lk$.
    
    L'application $\varphi$ est un homomorphisme de groupes. En effet, pour
    tout $(k,l)\in K\times L$ et $(k',l')\in K\times L$, on a
    \[
      \varphi(k,l)\varphi(k',l')
      = klk'l'=kk'll'
      = \varphi(kk',ll')
      = \varphi((k,l)(k',l')).
    \]
    Par construction, $\varphi$ est clairement surjectif. Enfin, soit
    $(k,l)\in K\times L$ tel que $\varphi(k,l) = e$, c'est-à-dire $kl = e$.
    Alors $k = l^{-1}\in K\cap L$ et $l = k^{-1}\in K\cap L$, si bien que $(k,l) = (e,e)$;
    l'homomorphisme $\varphi$ est injectif.

    En résumé, le sous-groupe $KL$ est isomorphe au groupe $K\times L$.

  \item 
    Soit $H$ un sous-groupe normal de $\Z$ différent de $(0)$.  Il existe
    $n\in\N^*$ tel que $H = n\Z$. Soit $H' = 2n\Z$. Alors $H'\normal \Z$ et
    $H'<H$ avec $H'\neq (0)$. Donc $H$ n'est pas minimal. On conclut que
    $(\Z,+)$ n'a pas de sous-groupe normal minimal.
\end{enumerate}

